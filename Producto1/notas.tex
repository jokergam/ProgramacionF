%Manual de comandos de bash en Linux

%Especifico tipo de documento y el tamaño de letra
\documentclass[notitlepage,12pt]{article}

%El documento estara en español, usar paquete en español
\usepackage[spanish]{babel}
\selectlanguage{spanish}
\usepackage[utf8]{inputenc}
%Permite colores
\usepackage{color}
%topmatter (titulo, autor, fecha)
\title{Manual de comandos de bash para Linux}
\author{Hugo Valenzuela Chaparro}
\date{\today}



%Inicio del cuerpo del documento
\begin{document}
\begin{abstract}
\color{blue} Existe otra manera de darle \'ordenes a una computadora sin tener que estar usando el mouse y haciendo muchos clicks, dicha manera es escribiendo e intruduciendo las instrucciones (comandos) con el teclado. Este documento es un manual de comandos de bash, para Linux. La estructura ser\'a comando, descripci\'on y ejemplo.
\end{abstract}



\maketitle

%Definicion
\section{Bash}
Bash es un intermediario entre el usuario y el sistema operativo, {\color{red} Linux} en este caso. Ejecuta la acci\'on de los comandos que el usuario introduce.

\section{Objetivos}
\begin{itemize}
\item Mostrar comandos b\'asicos de Linux.
\item Servir de consulta, por si se necesita recordar un comando en espec\'ifico.
\item Aclarar la funci\'on de los mencionados comandos, con ejemplos.
\end{itemize}

\section{Comandos}

%Lista de comandos 
\begin{tabular}{|p{2cm}|p{8cm}|p{5cm}|}
\hline
Comando & Descripci\'on & Ejemplo \\
\hline
man & Comando b\'asico para ver las funciones de los mismos comandos & man ls \\
\hline
mkdir & Crea una carpeta seg\'un el nombre que le pongas
 &  mkdir CarpetaNueva \\
\hline
rmdir &  Borra una carpeta vac\'ia &
 rmdir CarpetaNueva \\
\hline
cd & Permite entrar a las distintas carpetas &  cd Documentos, cd /home/Hugo/Juegos/Xbox \\
\hline
pwd & Nos dice la direcci\'on del directorio en el que nos encontramos & \\
\hline
ls & Muestra el contenido de una carpeta & \\
\hline
ls -l & Muestra el contenido de la carpeta, pero de manera m\'as detallada (permisos, tamaño, \'ultima fecha de modificaci\'on, etc) & \\
\hline
file & Nos dice el tipo de archivo & cd Canci\'on \\
\hline
cp & Copia un archivo o un directorio, pudiendo darle un nombre nuevo & cp Fender Squier \\
\hline
mv & Mueve directorios o archivos, pudiendo darle nombre nuevo & mv c1 programas, mv c1 CarpetaNueva/NombreNuevodec1 \\
\hline
rm & Borra archivos o carpetas no vac\'ias & rm horario, rm Juegos \\
\hline
touch & Crea un archivo en blanco & touch readme \\
\hline
vi & Editor de texto, crea archivos de texto o edita & vi CancionNueva \\
\hline
cat & Ver el contenido de un archivo de texto (si es largo s\'olo se ve la \'ultima p\'agina) & cat CancionNueva \\
\hline
less & Ver el contenido de un archivo (sirve para archivos largos, navegamos con las flechas) & less CancionNueva \\
\hline
chmod & Cambia permisos, quita (+) o permite (-), usuario (u), grupo(g) u otros (o) & chmod u+r CancionNueva \\
\hline
ls -ld & Ver permisos de una carpeta & ls -ld CarpetaNueva \\
\hline
history & Muestra el historial de lo que se ha estado haciendo & \\
\hline
clear & Quita el texto de lo que has estado haciendo en la terminal & \\
\hline
\end{tabular} 

\end{document}

