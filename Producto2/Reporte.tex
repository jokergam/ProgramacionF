%Reporte sobre compiladores e interpretadores, y lenguajes de programacion
\documentclass[notitlepage,12pt]{article}

%El documento estara en español, usar paquete en español
\usepackage[spanish]{babel}
\selectlanguage{spanish}
\usepackage[utf8]{inputenc}
%Permite colores
\usepackage{color}
%topmatter (titulo, autor, fecha)
\title{Lenguajes de programaci\'on}
\author{Hugo de Jes\'us Valenzuela Chaparro}
\date{\today}

\begin{document}
\maketitle

%Resumen
\section{Compiladores e interpretadores}
La funci\'on de los {\color{blue} interpretadores} es de interpretar, literalmente, es decir, ejecutan un programa directamente del c\'odigo sin necesidad de hacer un archivo ejecutable. Por otro lado, los {\color{red} compiladores} lo que hacen es traducir el c\'odigo a el lenguaje m\'aquina, creando un ejecutable el cual la m\'aquina ejecutar\'a. Por esa raz\'on cuando se modifica el c\'odigo de un programa y no se compila, el ejecutable sigue con la misma func\'on, por su parte, cuando se modifica el de c\'odigo de un interpretador, cambia la funci\'on a lo que se modific\'o.

\section{Lenguajes de programaci\'on}
\begin{tabular}{|p{2cm}|p{3cm}|p{2cm}|p{2cm}|p{1cm}|}
\hline
{\bf{Nombre}} & {\bf{Paradigma}} & {\bf{Creadores}} & {\bf{Año de aparici\'on}} & {\bf{Ext. de archivo}} \\
\hline
ANSI C & Imperativo  & Dennis Ritchie  & 1972  & .c \\
\hline
C++ & Orientado a objetos  & Bjarne Stroustrup & 1983  & .cpp \\
\hline
Fortran 90 & Orientado a objetos & John Backus & 1990 & .f90 \\
\hline
Java & Orientado a objetos & James Gosling y Sun Microsystems & 1995 & .java \\
\hline
Python & Orientado a objetos & Guido van Rossum & 1991 & .py \\
\hline
Ruby & Orientado a objetos & Yukihiro Matsumoto & 1995 & .rb \\
\hline

\end{tabular} 

\section{Ejemplos de c\'odigos}
\subsection{ANSI C}
\begin{verbatim}
#include <stdio.h>

int main()
{
printf("Hola! Tratare de adivinar un numero. 
Piensa en un numero entre 1 y 10\n");
sleep(5);
printf("Ahora multiplicalo por 9\n");
sleep(5);
printf("Si el numero tiene 2 digitos, sumalos entre si.
Si tu numero tiene un solo digito, sumale 0\n");
sleep(5);
printf("Al numero restante sumale 4\n");
sleep(10);
printf("Muy bien. El resultado es 13 ;D. 
Siguele rockeando con todo, animo.\n");
return 0;
}
\end{verbatim}
\subsection{C++}
\begin{verbatim}
#include <iostream>
#include <unistd.h>
using namespace std;

int main()
{
cout << "Hola! Tratare de adivinar un numero. 
Piensa en un numero entre 1 y 10" << endl; 
sleep(5);
cout << "Ahora multiplicalo por 9" << endl;
sleep(5);
cout << "Si el numero tiene 2 digitos, sumalos entre si. 
Si tu numero tiene un solo digito, sumale 0" << endl;
sleep(5);
cout << "Al numero restante sumale 4" << endl;
sleep(10);
cout << "Muy bien. El resultado es 13 ;D. 
Siguele rockeando con todo, animo." << endl;
return 0;
}
\end{verbatim}
\subsection{Fortran 90}
\begin{verbatim}
PROGRAM AdivinaLaMente

PRINT*, "¡Hola! Tratare de adivinar un numero. &
&Piensa en un numero entre 1 y 10"
CALL Sleep (5)
PRINT*, "Ahora multiplicalo por 9"
CALL Sleep (5)
PRINT*, "Si el numero tiene 2 digitos, sumalos entre si. Si tu numero tiene&
& un solo digito, sumale 0"
CALL Sleep (5)
PRINT*, "Al numero restante sumale 4"
CALL Sleep (10)
PRINT*, "Muy bien. El resultado es 13 ;D. Siguele rockeando con todo, animo."

ENDPROGRAM AdivinaLaMente
\end{verbatim}
\subsection{Java}
\begin{verbatim}
public class AdivinaLaMente{

     public static void main(String []args){
	 try  {
System.out.println("Hola! Tratare de adivinar un numero. 
Piensa en un numero entre 1 y 10");
Thread.sleep(5000);
System.out.println("Ahora multiplicalo por 9");
Thread.sleep(5000);
System.out.println("Si el numero tiene 2 digitos, sumalos entre si. 
Si tu numero tiene un solo digito, sumale 0");
Thread.sleep(5000);
System.out.println("Al numero restante sumale 4")
Thread.sleep(10000)
System.out.println("Muy bien. El resultado es 13 ;D. 
Siguele rockeando con todo, animo.");



     }
	 catch(Exception e) {
	 }
}

}
\end{verbatim}
\subsection{Python}
\begin{verbatim}
#!/usr/bin/python
import time
     
print "Hola! Tratare de adivinar un numero. Piensa en un numero entre 1 y 10"
time.sleep( 5 )
print "Ahora multiplicalo por 9"
time.sleep( 5 )
print "Si el numero tiene 2 digitos, sumalos entre si. 
Si tu numero tiene un solo digito, sumale 0"
time.sleep( 5 )
print "Al numero restante sumale 4"
time.sleep( 10 )
print "Muy bien. El resultado es 13 ;D. Siguele rockeando con todo, animo."
\end{verbatim}
\subsection{Ruby}
\begin{verbatim}
#Adivina la mente en ruby

puts "Hola! Tratare de adivinar un numero. Piensa en un numero entre 1 y 10!"
sleep(5)
puts "Ahora multiplicalo por 9"
sleep(5)
puts "Si el numero tiene 2 digitos, sumalos entre si.
Si tu numero tiene un solo digito, sumale 0"
sleep(5)
puts "Al numero restante sumale 4"
sleep(10)
puts "Muy bien. El resultado es 13 ;D. Siguele rockeando con todo, animo."
\end{verbatim}


\end{document}
