%Reporte sobre los programas del produto 3
\documentclass[notitlepage,12pt]{article}

%El documento estara en español, usar paquete en español
\usepackage[spanish]{babel}
\selectlanguage{spanish}
\usepackage[utf8]{inputenc}
%Permite colores
%Permite incorporar imagenes
\usepackage{graphicx}
\usepackage{color}
%topmatter (titulo, autor, fecha)
\title{Iniciando con Fortran}
\author{Hugo de Jes\'us Valenzuela Chaparro}
\date{\today}

\begin{document}
\maketitle

\section{Area de un circulo}
Con este programa se calcula el \'area de un c\'irculo de radio R que el usuario indica.
Aqu\'i un ejemplo:
\begin{verbatim}
! Area . f90 : Calcula el area de un circulo

 ! −−−−−−−−−−−−−−−−−−−−−−−−−−−−−−−−−−−−−−−−−−−−−−−

 Program Circle_area ! Comenzar programa

   Implicit None ! Declaracion de variables

   Real  :: radius , circum , area ! Declarar reales

   Real,PARAMETER  :: PI = 4.0 * atan(1.0) ! Declarar constante real

   Integer :: model_n = 1 ! Declare , assign Ints

   print*, "Escribe el radio del circulo:" ! hablar al usuario

   read*, radius ! leer radio

  circum = 2.0*PI*radius ! calcular circunferencia

  area = radius*radius*PI ! calcular

  print*, "Numero de programa =" , model_n ! Print program number

  print*, "Radio =" , radius ! imprimir radio

  print*, "El perimetro es =" , circum , "unidades" ! imprimir perimetro

  print*, "Area =" , area , "unidades cuadradas" ! imrpimir area
 End Program Circle_area ! Terminar programa
\end{verbatim}
A contunuaci\'on una imagen del programa corriendo:



\includegraphics[scale=0.5]{1_xAreaCirculo}

\section{Volumen tanque esf\'erico}
En este programa se calcul\'o el volumen de agua en un tanque esf\'erico de radio R,
de acuerdo a la altura que este el agua respecto al fondo. Ambos datos fueron introducidos
por el usuario.
Aqu\'i el c\'odigo:
\begin{verbatim}
! Volumen.f90 : Calcula el volumen del agua que se encuentra 
! en un tanque esferico a una altura h del suelo
 ! −−−−−−−−−−−−−−−−−−−−−−−−−−−−−−−−−−−−−−−−−−−−−−−

 Program Volumen_esfera ! comenzar pograma

   Implicit None ! Declaracion de variables

   Real  :: radio , volumen , altura   ! Declarar reales

   Real,PARAMETER  :: PI = 4.0 * atan(1.0) ! Declar constaste

   Integer :: model_n = 2 ! Declarar entero con valor dado

   
   !Pedir datos
   ! hablar al usuario
   print*, "Escribe el radio de la esfera (mayor o igual que cero):" o
   read*, radio
   print*, "Escribe la altura a la que esta el agua desde el piso &
   & o fondo del tanque esferico"   
   read*, altura

   IF (altura <= (2.0*radio)) THEN
     ! calcular el volumen del agua en el tanque   
     volumen = (PI/3.0)*(altura*altura)*((3.0*radio)-altura)
   
   ELSE
       print*, "Verifica que los datos sean consistentes"
        STOP
   END IF
  


  print*, "Numero de programa =" , model_n ! Print program number
  print*, "El radio del tanque esferico es" , radio !radio
  print*, "La altura del agua respecto al fondo del tanque es" , altura !altura
  print*, "El volumen que se encuentra en el tanque esferico a &
  & esa altura es =" , volumen , "unidades cubicas" 

 End Program Volumen_esfera ! Terminar programa
\end{verbatim}
Imagen del programa corriendo:



\includegraphics[scale=0.5]{2_xVolumenTanque}

\section{Presici\'on Sencilla *4}
Este programa sirve para ver la presici\'on sencilla *4
de la m\'aquina en los c\'alculos.
Aqu\'i el c\'odigo:
\begin{verbatim}
! Limits . f90 : Determina la presicion de la maquina, 
 ! presicion sencilla *4


 ! −−−−−−−−−−−−−−−−−−−−−−−−−−−−−

 Program Limits

   Implicit None

   Integer :: i , n

   Real *4 :: epsilon_m , one
   Integer :: model_n = 41 ! Declare , assign Ints
   n=60 ! Establece el numero de iteraciones

   ! Dar valores iniciales :

   epsilon_m = 1.0

  one = 1.0

  ! calcular cada paso con DO-LOOP e imprimir

  !  Se ejecutara 60 veces de acuerdo a i

  !  Incrementado de 1 a n (debido a n=60)
  print*, "Numero de programa =" , model_n !Print program number
  do i = 1, n , 1 ! Comienza el loop

    epsilon_m = epsilon_m / 2.0 ! reduce epsilon m

    one = 1.0 + epsilon_m !  calcula de nuevo one

    print*, i , one , epsilon_m !  imprimir valores

  end do ! terminar loop cuando  i>n

 End Program Limits 
\end{verbatim}
Imagen del programa corriendo:

\includegraphics[scale=0.4]{4_1_xPresicionSencilla_4}

\section{Presici\'on sencilla real}
En este programa se ve la presici\'on sencilla
real de la m\'aquina.
Aqu\'i el c\'odigo:
\begin{verbatim}
! Limits . f90 : Determina la presicion de la maquina, 
 ! presicion sencilla *4


 ! −−−−−−−−−−−−−−−−−−−−−−−−−−−−−

 Program Limits

   Implicit None

   Integer :: i , n

   Real  :: epsilon_m , one
   Integer :: model_n = 42 ! Declare , assign Ints
   n=60 ! Establece el numero de iteraciones

   ! Dar valores iniciales :

   epsilon_m = 1.0

  one = 1.0

  ! calcular cada paso con DO-LOOP e imprimir

  !  Se ejecutara 60 veces de acuerdo a i

  !  Incrementado de 1 a n (debido a n=60)
  print*, "Numero de programa =" , model_n ! Print program number
  do i = 1, n , 1 ! Comienza el loop

    epsilon_m = epsilon_m / 2.0 ! reduce epsilon m

    one = 1.0 + epsilon_m !  calcula de nuevo one

    print*, i , one , epsilon_m !  imprimir valores

  end do ! terminar loop cuando  i>n

 End Program Limits 
\end{verbatim}
Imagen del programa corriendo:

\includegraphics[scale=0.4]{4_2_xPresicionSencilla_Real}

\section{Funciones instr\'insecas}
Este programa sirve para ejemplificar c\'omo y
para qu\'e se usan las funciones instr\'insecas
de Fortran.
Aqu\'i el c\'odigo:
\begin{verbatim}
! Math . f90 : ejemplos de algunas funciones de Fortran

 ! −−−−−−−−−−−−−−−−−−−−−−−−−−−−−−−−−−−−−−−−−−

 Program math ! Comenzar programa
    ! Declaracion de variables4   
    Real *8 :: x =-1.0 , y=0.2 , z=0 , arccos, log, i, a
    Integer :: model_n = 6 ! Declare , assign Ints
   
   a=ABS(x)
   i =SQRT(a)
   arccos=ACOS(y)
   
   
   
   print*, "Numero de programa =" , model_n !Print program number
   print*, "La raiz cuadrada de -1 es: +-" , i , "i"
   print*, "El arcoseno de 0.2 es:" , arccos
   print*, "El logaritmo de 0 no esta definido"
   
 End Program math ! Terminar programa
\end{verbatim}
Imagen del programa corriendo:

\includegraphics[scale=0.5]{6_xmath}

\section{Funciones definidas por el usuario}
Este programa sirve para ilustrar c\'omo se definen funciones
en Fortran y c\'omo se llaman en el programa principal.
Aqu\'i el c\'odigo:
\begin{verbatim}
! Function . f90 : Llama a una funcion definida por el usuario

 ! −−−−−−−−−−−−−−−−−−−−−−−−−−−−−−−

 Real *8 Function f (x,y)

   Implicit None

   Real *8 :: x, y

   f = 1.0 + sin (x*y )

 End Function f

 

 Program Main

  Implicit None

  Real *8 :: Xin =0.25 , Yin =2. , c , f ! declarations ( also f)
  Integer :: model_n = 7 ! Declare , assign Ints
  c = f ( Xin , Yin )
  print*, "Numero de programa =" , model_n ! Print program number
  write ( * , * ) "f(Xin, Yin) = " , c

 End Program Main 
\end{verbatim}
Imagen del programa corriendo:

\includegraphics[scale=0.5]{7_xfuncion}

\section{Subrutinas}
Este programa ilustrar c\'omo se hace una subrutina
y se llama en el programa principal.
Aqu\'i en c\'odigo:
\begin{verbatim}
! Subroutine . f90 : Muestra como se llama una subrutina
 ! −−−−−−−−−−−−−−−−−−−−−−−−−−−−−−−−−−−−−−−−−−−−−

 Subroutine g(x, y, ans1 , ans2 )

   Implicit None

   Real (8) :: x , y , ans1 , ans2 ! Declarar variables

   ans1 = sin (x*y) + 1. ! Usar funcion instrinseca

   ans2 = ans1**2

 End Subroutine g

 !

 Program Main ! Demos the CALL

   Implicit None

   Real *8 :: Xin =0.25 , Yin =2.0 , Gout1 , Gout2
   Integer :: model_n = 8 ! Declare , assign Ints
   call g( Xin , Yin , Gout1 , Gout2 ) ! Llamar la subrutina
   
    print*, "Numero de programa =" , model_n !Print program number

   write ( * , *) "Las respuestas son: " , Gout1 , Gout2

 End Program Main
\end{verbatim}
Imagen del programa corriendo:

\includegraphics[scale=0.5]{8_xsubrutina}






 \end{document}
